\documentclass[10pt,a4paper]{article}
\usepackage[utf8]{inputenc}
\usepackage[german]{babel}
\usepackage[T1]{fontenc}
\usepackage{amsmath}
\usepackage{amsfonts}
\usepackage{amssymb}

\author{Florian Grimm}
\title{CS102 \LaTeX \hspace{1pt} Übung}
\date{06. November 2014}

\begin{document}

\maketitle

\section{Arbeiten mit \LaTeX}

Das Arbeiten mit LaTeX  ist anfangs verwirrend, wird aber mit ein bisschen Übung zum Kinderspiel.

\section{Tabelle}

Unsere wichtigsten Daten finden Sie in Tabelle 1.

\begin{table}[htpb]
\centering
\begin{tabular}{l|c|c|c}
\newline & Punkte erhalten & Punkte möglich & \% \\
\hline Aufgabe 1 & 2 & 4 & 0.5  \\
\hline Aufgabe 2 & 3 & 3 & 1.0  \\
\hline Aufgabe 3 & 3 & 3 & 1.0
 
\end{tabular}
\caption{Aufgabe 2}
\end{table}
\section{Formeln}
\subsection{Pythagoras}

Der Satz des Pythagoras errechnet sich wie folgt:
$a^2+b^2=c^2$. Daraus können wir die Länge der Hypothenuse wie folgt berechnen:
$c=\sqrt{a^2+b^2}$.
\subsection{Summen}
Wir können auch die Formel für eine Summe angeben:
\begin{equation}
s=\sum_{i=1}^n \ i=\frac{n*(n+1)}{2}
\end{equation}
\end{document}
#Hallo
